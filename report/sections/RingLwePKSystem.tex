%! Author = sellebjergen
%! Date = 25/02/2022

% TODO: Måske der skal indsættes et eksempel af de her 2? Burde være ganske let at gøre, da vi kan bruge implementationen

% Preamble
\documentclass[11pt]{article}

% Packages
\usepackage{amsmath}
\usepackage{amsfonts}

% Document
\begin{document}
    \section*{Encoding messages as polynomials}
    Encoding messages as polynomials follows simply by the fact that a letter can be represented by a number.
    These numbers can then be combined into a vector which can be encoded as a polynomial.
    As an example take the message: ``CRYPTO'', using the ASCII table, this message can be represented by the
    vector $[67, 82, 89, 80, 84, 79]$.
    This vector can be represented as a polynomial by $67 + 82x + 89x^2 + 80x^^ + 84x^4 + 79x^5$.
    Trivially we can also decode any polynomial into a message, by reverting these steps one by one.

    \section*{Ring LWE}
    In the paper ``Fully Homomorphic Encryption from Ring-LWE and Security for Key Dependent Messages'' written by
    Zvika Brakerski and Vinod Vaikuntanathan \cite{} % TODO: få lige indsat ref
    they describe a method for converting the Ring Learnig with errors (RingLWE) problem into an encryption scheme
    which reduces to the worst-case hardness of problems on ideal lattices.
    We'll shortly describe the encryption scheme here but will omit proofs and detailed discussions.
    Both system will be over the message space of $R_t = \mathbb{Z}_t[x] / \langle f(x) \rangle $.

    \subsection*{Symmetric version}
    Let $\kappa$ be the security parameter and let further q and t be prime numbers where $t \in \mathbb{Z}_n^*$.
    We also need a polynomial of degree n $f(x) \in \mathbb{Z}[x]$ and an error distribution $\chi$ over the ring
    $R_q = \mathbb{Z}_q[x] / \langle f(x) \rangle$.

    \subsubsection*{Key-gen}
    Given the security parameter $\kappa$ sample a ring element s uniformly at random from $\kappa$ and define the
    secret key vector as $(s^0, s^1, s^2, \dots, s^D) \in R_q^{D+1}.

    \subsubsection*{Encryption}
    \subsubsection*{Eval}
    \subsubsection*{Decryption}

    \subsection*{Public key version}
        hejsa

\end{document}