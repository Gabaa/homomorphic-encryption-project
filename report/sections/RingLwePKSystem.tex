% !TeX root = ../main.tex

\documentclass[../main.tex]{subfiles}
\graphicspath{{\subfix{../images/}}}

\lstdefinelanguage{Sage}[]{Python}
{morekeywords={False,sage,True},sensitive=true}
\lstset{
  frame=none,
  showtabs=False,
  showspaces=False,
  showstringspaces=False,
  commentstyle={\ttfamily\color{dgreencolor}},
  keywordstyle={\ttfamily\color{dbluecolor}\bfseries},
  stringstyle={\ttfamily\color{dgraycolor}\bfseries},
  language=Sage,
  basicstyle={\fontsize{10pt}{10pt}\ttfamily},
  aboveskip=0.3em,
  belowskip=0.1em,
  numbers=left,
  numberstyle=\footnotesize
}
\definecolor{dblackcolor}{rgb}{0.0,0.0,0.0}
\definecolor{dbluecolor}{rgb}{0.01,0.02,0.7}
\definecolor{dgreencolor}{rgb}{0.2,0.4,0.0}
\definecolor{dgraycolor}{rgb}{0.30,0.3,0.30}
\newcommand{\dblue}{\color{dbluecolor}\bf}
\newcommand{\dred}{\color{dredcolor}\bf}
\newcommand{\dblack}{\color{dblackcolor}\bf}

% Document
\begin{document}

\section{Ring-LWE - A somewhat-homomorphic encryption scheme} \label{sec:RLWE-theory}

In the paper ``Fully Homomorphic Encryption from Ring-LWE and Security for Key Dependent Messages'' written by
Zvika Brakerski and Vinod Vaikuntanathan \cite{brakerski2011fully}
they describe a method for converting the Ring Learning with errors (Ring-LWE) problem into an encryption scheme,
which reduces to the worst-case hardness of problems on ideal lattices.
We will shortly describe the encryption scheme here, but will omit proofs and detailed discussions.
% TODO [meg]: Her skal vi nok forklare hvilke "systems" vi taler om.
% TODO [meg]: Indså også lige at det er $R_p$, men vi har ikke defineret $p$ endnu. Bedst at nævne senere?
Both systems will be over the message space of $R_p = \mathbb{Z}_p[x] / \langle f(x) \rangle$.

% TODO [meg]: Skriver vi om RLWE eller PLWE? Vi bør nok skrive noget om sammenhængen mellem de to, og gøre "overgangen" (fra RLWE til PLWE) lidt mere tydelig.

\subsection{Polynomial learning with errors}

The decisional polynomial learning with errors (PLWE) problem is parameterized by a polynomial $f(x) \in \mathbb{Z}[x]$ where $deg(f) = N$, a prime $q \in \mathbb{Z}$, a distribution $\chi$ (specifically an error distribution over elements in $R_q$), and an integer $\ell$ (a limit on the number of samples given in the problem).
$f$ defines the ring $R = \mathbb{Z}[x] / \langle f(x) \rangle$, and $R$ and $q$ together define $R_q = R / q R = \mathbb{Z}_q[x] / \langle f(x) \rangle$.
An instance of the problem is then written as $PLWE_{f, q, \chi}^{(\ell)}$.

% TODO [meg]: Er ikke helt tilfreds med at denne er skrevet som assumption når vi bruger problem over, men det kan vi lige snakke om.
\begin{definition}[The $PLWE_{f, q, \chi}^{(\ell)}$ assumption]
    Let $s$ be a uniformly random element from $R_q$.
    Then it holds that 
    \[
        \{ (a_i, a_i \cdot s + e_i) \}_{i \in [\ell]} \approx^{c} \{(a_i, u_i) \}_{i \in [\ell]}\}
    \]
    where all $a_i$ and $u_i$ are uniformly random elements from $R_q$, and all $e_i$ are sampled from $\chi$.
\end{definition}

%%% Den gamle formulering af PLWE. Skal den bare slettes?
% 
% The polynomial learning with errors problem decision problem variant is defined for an instantiation
% of a function $f$, a modulo $q$ and an error distribution $\chi$ and given at most $\ell$ samples written as $PLWE_{f, q, \chi}^{(\ell)}$.
% It is defined in the hermite normal form as
% \\[2mm]
% \textbf{The PLWE Assumption - Hermite Normal Form.}
% For all $\kappa \in \mathbb{N}$,let $f(x) = f_{\kappa}(x) \in \mathbb{Z}[x]$
% be a polynomial of degree $N = N(\kappa)$, let $q = q(\kappa) \in \mathbb{Z}$
% be a prime integer, let the ring $R = \mathbb{Z}[x] / \langle f(x) \rangle$
% and $R_q = R / qR$ with $\chi$ denoting a distribution over the ring R.
% Then the Polynomial learning with error ($PLWE_{f,q,\chi}$) assumption can be defined as
% \[
%     \{ (a_i, a_i \cdot s + e_i) \}_{i \in [\ell]} \approx^{c} \{(a_i, u_i) \}_{i \in [\ell]}\}
% \]
% where s is sampled from $\chi$, $a_i$ is uniform in $R_q$, the error polynomials
% $e_i$ are sampled from $\chi$ and $u_i$ are random ring elements from $R_q$.
% We will allow up to $\ell$ samples for the decision problem where the only limitation on $\ell$ is that it should
% be polynomial.
% \\[5mm]

Correspondingly, the decisional $PLWE_{f, q, \chi}^{(\ell)}$ problem is then to distinguish how a set of $\ell$ samples was generated: as elements of the form $(a_i, a_i \cdot s + e_i)$, or of the form $(a_i, u_i)$.

% TODO [meg]: Kan ikke finde ud af om dette skal være her eller ej ¯\_(ツ)_/¯
Beside the decision variant, the PLWE problem could also be stated in terms of a search problem, one in which
an adversary has to find the secret vector $s \in R_q$. But we will focus on the decision problem, as this
is more commonly used in cryptography.

\subsection{Symmetric encryption scheme}

Let $\kappa$ be the security parameter and let further $p$ and $q$ be prime numbers where $p \in \mathbb{Z}_N^*$.
We also need a polynomial of degree $N$: $f(x) \in \mathbb{Z}[x]$ and an error distribution $\chi$ over the ring
$R_q = \mathbb{Z}_q[x] / \langle f(x) \rangle$, then we can define the following operations for a somewhat homomorphic
encryption scheme.

\subsubsection{Key generation}
Let our secret key be a randomly sampled element from the error distribution $s \leftarrow^\$ \chi$.
Then, given the security parameter $\kappa$, sample a ring element $s$ uniformly at random from $\kappa$ and define the
secret key vector by $(s^0, s^1, s^2, \dots, s^D) \in R_q^{D+1}$.

\subsubsection{Encryption}
As mentioned, all messages are encodeable in our message space $R_p$, thus we will encode our message $m$ as a degree $N$
polynomial with coefficients modulo $p$.
To encrypt, we sample $(a, b = a \cdot s + p \cdot e)$ where $a \leftarrow^\$ R_q$ and $e \leftarrow^\$ \chi$,
then compute
\begin{align*}
    c_0 \coloneqq b + m &  & c_1 \coloneqq -a
\end{align*}
and from this output the ciphertext \textbf{c} $\coloneqq (c_0, c_1) \in R_q^2$.

\subsubsection{Decryption}
% TODO [meg]: Skulle vi nævne noget med l inf norm kravet her? Vi refererer til det i Circuit Privacy.
Note that a ciphertext is on the form $(c_0, c_1, \dots , C_{D}) \in R_q^{D+1}$.
Define the inner product over $R_q$ as
\begin{align*}
    \langle c, s \rangle = \sum_{i=0}^{D} c_i \cdot s^i
\end{align*}
Then to decrypt, simply set $m$ as the inner product of $c$ and $s$ and take modulo $p$.
\begin{align*}
    m = \langle c, s \rangle \bmod p
\end{align*}
$m$ will then be the decrypted message.

\subsubsection{Eval}
To obtain the homomorphic abilities of the encryption scheme, Zvika Brakerski and Vinod Vaikuntanathan show
how to obtain homomorphic addition and multiplication of ciphertexts.
\\[2mm]
\textbf{Addition:} Assume we have 2 ciphertexts $c \in R_q^{D+1}$ and $c' \in R_q^{D+1}$, then an encryption
of the sum of the 2 underlying messages will be
\begin{align*}
    c_{Add} = c + c' = (c_0 + c'_0, c_1 + c'_1, \dots , c_d + c'_d) &  & c_{Add} \in R_q^{D + 1}
\end{align*}
The decryption of $c_{Add}$ will then be the sum of the unencrypted messages from $c$ and $c'$.
\\[2mm]
\textbf{Multiplication:} Assume we have 2 ciphertexts $c \in R_q^{D+1}$ and $c' \in R_q^{D'+1}$ and let $v$ be a symbolic value.
Then calculate the updated ciphertext
\\
$(\hat{c}_0, \hat{c}_1, \dots, \hat{c}_d+d') \in R_q^{D + D' + 1}$ by
\begin{align*}
    c_{mul} = (\sum_{i=0}^D c_i \cdot v^i) \cdot
    (\sum_{j=0}^{D'} c'_i \cdot v^i) =
    \sum_{i=0}^{D+D'} \hat{c}_i \cdot v^i
     &  & c_{mul} \in R_q^{D+D'+1}
\end{align*}
The output of the multiplication operation will then be $c_{mul} = (\hat{c}_0, \hat{c}_1, \dots, \hat{c}_{D+D'})$

\subsection{Public-key encryption scheme}
%TODO: måske der skal skrives noget om lemma 4?
To achieve a public scheme instead, we can make the following changes
\begin{itemize}
    \item In the key generation we generate in addition to the secret key $sk = s \leftarrow^{\$} \chi$, a public key
        $pk = (a_0 , b_0 = a_0 \cdot s + p \cdot e_0)$, where $a_0 \leftarrow^{\$} R_q, e_0 \leftarrow^{\$} \chi$
    \item In the encryption algorithm, we can instead use
        $(a_0 \cdot v + t \cdot e', b_0 \cdot v + p \cdot e'')$ where
        $v, e' \leftarrow^{\$} \chi$ and $e'' \leftarrow^{\$} \chi '$. % TODO [meg]: Vi bør forklare hvad $\chi '$ er.
\end{itemize} 
where we have then obtained a public key $pk = (a_0, a_0 \cdot s + p \cdot e_0)$
corresponding to the secret key sk.

\subsection{On the security of PLWE}
    To later generate a secure set of parameters for our encryption scheme we will have to touch a bit on the security.
    Several considerations comes to play when thinking about the security and optimizations, most noticeable the choice
    of a proper function $f(x)$ and the best well known attack should come into consideration.
    A lot of math including calculations on lattices and especially ideal lattices are needed to properly discuss the
    necessity for certain choices to be made, and as such we will only be outlining the processes needed for us to choose
    secure parameters for our system.

    \subsubsection{On the choice of functions $f(x)$}
        The choice of function $f(x)$ is normally set to be a cyclotomic-polynomial, meaning that it should be on the form
        of $f(x) = x^N + 1$ where n should be chosen as $N = 2^k$ for $k \in \mathbb{N}$, do note that this is not the only
        way that a polynomial can be chosen as cyclotomic, but for our use case this will suffice.
        By doing so we get a lot of algebraic properties which opens for optimizations as described in section % TODO: sæt lige ref.
        and adds to the security.
        Therefore we will without further discussion choose $f(x)$ as a cyclotomic-polynomial function.

    \subsubsection{Reducing RLWE to LWE samples}
        Another important property to see is that any RLWE sample of the form $(a, s \cdot a + e)$,
		where $a \in R_q$ and $s, e \leftarrow^\$ \chi$ can be written into N LWE samples by using the following method.
        Let $A_a$ be the matrix of multiplication by $a \in R_q$, then we get N LWE samples by
        \begin{align*}
            (A_a, \textbf{b} = \textbf{s}^T \cdot A_a + \textbf{e}^T)
        \end{align*}
        Where $\textbf{b} \in (\mathbb{R} / q\mathbb{Z})^N$ and $e \in \mathbb{R}^N$.

    \subsubsection{Parameter testing tool}
        Martin Albrecht, professor at the university of Royal Holloway in London, is the current maintainer of an
		opensource project~\cite{estimations-tool} geared towards choosing secure parameters for the RLWE encryption scheme.
		By using this tool, we will be able to calculate the amount of security by using the lattice reduction algorithm BKZ.
		We'll be using this tool to evaluate the amount of security that one would need to actually secure the algorithm,
		while still holding the degree of the function $f(x)$ to a minimum for performance reasons.
		The tool will use a lot of lattice based mathemathics to actually calculate the amount of security, which will be given
		to us in the form of a number $\beta$, that will then be used to test the amount of security.
        For the precise usage of the tool we have used the following sage script
        \lstinputlisting[language=Python]{code-snippets/estimator.sage}
        Then we can here vary the parameters of our modulus $q$, the degree of the cyclotomic-polynomial $N$
        our noise distributions on $r$ and $r'$ (Xs, Xe) and the value of $m$ typically provided by the authors.

    \subsubsection{Best known attack}
		The best known attack on the LWE scheme, which by combination of the earlier mentioned method to combine RLWE
		into $N$ LWE samples also gives an attack on RLWE, is the BKZ lattice reduction algorithm.
		We will not be going into how BKZ works, but will mention that the amount of security from the BKZ algorithm
		has been showed in the "new hope" paper \cite{usenix16} written by Erdem Alkim and Thomas poppelman to be
		\[ 2^{0.292 \cdot \beta} \]
		Where we can get the value of $\beta$ from the testing tool.

\end{document}
