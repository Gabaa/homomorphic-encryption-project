% !TeX root = ../main.tex
% TODO: Måske der skal indsættes et eksempel af de her 2? Burde være ganske let at gøre, da vi kan bruge implementationen

% Preamble
\documentclass[../main.tex]{subfiles}
\graphicspath{{\subfix{../images/}}}

% Document
\begin{document}

\section{Ring LWE - A somewhat homomorphic encryption scheme} \label{section: RLWE theory}
In the paper ``Fully Homomorphic Encryption from Ring-LWE and Security for Key Dependent Messages'' written by
Zvika Brakerski and Vinod Vaikuntanathan \cite{brakerski2011fully}
they describe a method for converting the Ring Learnig with errors (RingLWE) problem into an encryption scheme
which reduces to the worst-case hardness of problems on ideal lattices.
We'll shortly describe the encryption scheme here but will omit proofs and detailed discussions.
Both system will be over the message space of $R_t = \mathbb{Z}_t[x] / \langle f(x) \rangle $.

\subsection{Polynomial learning with errors}
The polynomial learning with errors problem decision problem variant is defined for an instantiation
of a function f, a modulo q and a error distribution $\chi$ and given at most l samples written as $PLWE_{f, q, \chi}^(l)$
It is defined in the hermite normal form as
\\[2mm]
\textbf{The PLWE Assumption - Hermite Normal Form.}
for all $\kappa \in \mathbb{N}$, let $f(x) = f_{\kappa}(x) \in \mathbb{Z}[x]$
be a polynomial of degree n = n($\kappa$), let $q = q(\kappa) \in \mathbb{Z}$
be a prime integer, let the ring $R = \mathbb{Z}[x] / \langle f(x) \rangle$
and $R_q = R / qR$ with $\chi$ denoting a distribution over the ring R.
Then the Polynomial learning with error ($PLWE_{f,q,\chi}$) assumption can be defined as
$$
    \{ (a_i, a_i \cdot s + e_i) \}_{i \in [l]} \approx^{c} \{(a_i, u_i) \}_{i \in [l]}\}
$$
where s is sampled from $\chi$, $a_i$ is uniform in $R_q$, the error polynomials
$e_i$ are sampled from $\chi$ and $u_i$ are random ring elements from $R_q$.
We will allow up to l samples for the decision problem where the only limitation on l is that it should
be polynomial.
\\[5mm]
Beside the decision variant, the PLWE problem could also be stated in terms of a search problem, one in which
an adversary have to find the secret vector $s \in R_q$. But we will focus on the decision problem, as this
is more commonly used in cryptography.

\subsection{Symmetric version}
Let $\kappa$ be the security parameter and let further q and t be prime numbers where $t \in \mathbb{Z}_n^*$.
We also need a polynomial of degree n $f(x) \in \mathbb{Z}[x]$ and an error distribution $\chi$ over the ring
$R_q = \mathbb{Z}_q[x] / \langle f(x) \rangle$, then we can define the following operations.

\subsubsection{Key-gen}
Let our secret key be a randomly sampled element from the error distribution $s \leftarrow^\$ \chi$
Then given the security parameter $\kappa$ sample a ring element s uniformly at random from $\kappa$ and define the
secret key vector by $(s^0, s^1, s^2, \dots, s^D) \in R_q^{D+1}$.

\subsubsection{Encryption}
Rememer that all messages are encodeable in our message space $R_t$, thus we will encode our message m as a n degree
polynomium with coefficient mod t.
To encrypt we sample $(a, b = a \cdot s + t \cdot e)$ where $a \leftarrow^\$ R_q$ and $e \leftarrow^\$ \chi$,
then compute
\begin{align*}
    c_0 \coloneqq b + m &  & c_1 \coloneqq -a
\end{align*}
and from this output the ciphertext \textbf{c} $\coloneqq (c_0, c_1) \in R_q^2$.

\subsubsection{Decryption}
Note that a ciphertext is on the form $(c_0, c_1, \dots , C_{D}) \in R_q^{D+1}$.
Define the inner product over $R_q$ as
\begin{align*}
    \langle c, s \rangle = \sum_{i=0}^{D} c_i \cdot s^i
\end{align*}
Then to decrypt, simply set m as the inner product of c and s and take modulo t.
\begin{align*}
    m = \langle c, s \rangle \bmod t
\end{align*}
m will then be the decrypted message.

\subsubsection{Eval}
To obtain the homomorphic abilities of the encryption scheme, Zvika Brakerski and Vinod Vaikuntanathan show
how to obtain homomorphic addition and multiplication of ciphertexts.
\\[2mm]
\textbf{Addition:} Assume we have 2 ciphertexts $c \in R_q^{D+1}$ and $c' \in R_q^{D+1}$, then an encryption
of the sum of the 2 underlying messages will be
\begin{align*}
    c_{Add} = c + c' = (c_0 + c'_0, c_1 + c'_1, \dots , c_d + c'_d) &  & c_{Add} \in R_q^{D + 1}
\end{align*}
The decryption of $c_{Add}$ will then be the sum of the unencrypted messages from $c$ and $c'$.
\\[2mm]
\textbf{Multiplication:} Assume we have 2 ciphertexts $c \in R_q^{D+1}$ and $c' \in R_q^{D'+1}$ and let v be a symbolig value
then calculate the updated ciphertext
\\
$(\hat{c}_0, \hat{c}_1, \dots, \hat{c}_d+d') \in R_q^{D + D' + 1}$ by
\begin{align*}
    c_{mul} = (\sum_{i=0}^D c_i \cdot v^i) \cdot
    (\sum_{j=0}^{D'} c'_i \cdot v^i) =
    \sum_{i=0}^{D+D'} \hat{c}_i \cdot v^i
     &  & c_{mul} \in R_q^{D+D'+1}
\end{align*}
The output of the multiplication operation will then be $c_{mul} = (\hat{c}_0, \hat{c}_1, \dots, \hat{c}_{D+D'})$

\subsection{Public key version}
%TODO: måske der skal skrives noget om lemma 4?
To achieve a public scheme instead, we can make the following changes
\begin{itemize}
    \item In the key generation we generate in addition to the secret key $sk = s \leftarrow^{\$} \chi$, a public key
        $pk = (a_0 , b_0 = a_0 \cdot s + t \cdot e_0)$, where $a_0 \leftarrow^{\$} R_q, e_0 \leftarrow^{\$} \chi$
    \item In the encryption algorithm, we can instead use
        $(a_0 \cdot v + t \cdot e', b_0 \cdot v + t \cdot e'')$ where
        $v, e' \leftarrow^{\$} \chi$ and $e'' \leftarrow^{\$} \chi '$.
\end{itemize} 
where we have then obtained a public key $pk = (a_0, a_0 \cdot s + t \cdot e_0)$
corresponding to the secret key sk.

\subsection{Choosing parameters and attacking the PLWE}
% hvert element i PLWE can også ses om n LWE elementer
% Disse kan løses med en algoritme der BKZ som giver en bit sikkerhed på 0.292 * \beta
% vi kan bruge toolet på https://notebooks.gesis.org/binder/jupyter/user/malb-lattice-estimator-u1sng1gm/notebooks/prompt.ipynb til at finde gode
%		data set at bruge.
One of the best known attacks against PLWE is BKZ written by 

\end{document}
