% !TeX root = ../main.tex
% Preamble
\documentclass[../main.tex]{subfiles}
\graphicspath{{\subfix{../images/}}}

\begin{document}

Secure multiparty computation is an important problem, where multiple parties can provide private inputs to securely evaluate a function and receive the output, without revealing any other information in the process.
One way of implementing such a system is with homomorphic encryption, which allows operations to be performed on encrypted data.

For this work, we implement a multiparty computation system based on somewhat homomorphism in the Rust programming language.
Our somewhat homomorphic encryption scheme is based on the polynomial learning with errors problem,
which is commonly used for post-quantum secure encryption schemes. 
The system is able to compute arithmetic formulas consisting of up to a single multiplication, along with a relatively large number of additions.
Supporting new arithmetic circuits is fairly simple by extending the code.

We found that in order for the system to be secure, the parameters must be prohibitively large relative to the performance of the system, making our implementation either insecure or impractical.
Specifically, running the system with secure parameters for a simple arithmetic circuit would require around $6.5$ days to compute, in large part due to an inefficient preprocessing phase.

\end{document}
