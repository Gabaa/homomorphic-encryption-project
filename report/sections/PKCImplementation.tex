% !TeX root = ../main.tex
\documentclass[../main.tex]{subfiles}
\graphicspath{{\subfix{../images/}}}

% Code environment (incomplete)
\lstset{
  basicstyle=\ttfamily,
}

\begin{document}

\section{Ring-LWE cryptosystem}

\subsection{Polynomials (poly.rs)}

Since our public-key encryption scheme uses polynomials to represent most of the values (messages, ciphertexts, secret keys, and public keys), we implemented a simple \lstinline{Polynomial} data structure, along with the most common operations we will perform on it.

Internally, a \lstinline{Polynomial} is simply a \lstinline{Vec} (a contiguous growable array type) of \lstinline{Integer} values, each representing a coefficient in the polynomial.

We implemented operations for adding, subtracting, negating, and multiplying (with both constants and other polynomials), and functions for trimming the polynomial (removing trailing zero-coefficients), right-shifting the coefficients (from lower to higher degrees), retrieving the $\ell_\infty$ value, calculating the modulo, and normalizing the coefficients to be in the range $[-q/2, q/2)$ instead of $[0, q)$, which is necessary during decryption.

\subsection{Quotient ring (quotient\_ring.rs)}

Encryption, decryption, and key generation involves adding, subtracting, multiplying, and negating elements in the quotient ring $R_q = \mathbb{Z}_q[x]/\langle x^N + 1 \rangle$. To be able to do these computations we made a quotient ring implementation, which can be found in \lstinline{quotient_ring.rs}.

\subsubsection{\lstinline{Rq}}

The quotient ring module contains a struct definition \lstinline{Rq}.
This struct represents an instantiation of a quotient ring $\mathbb{Z}_q[x]  \langle f(x) \rangle$.
It therefore has fields $q$ and $modulo$, where $q$ is an \lstinline{Integer}, and $modulo$ is a \lstinline{Polynomial} representing $f(x)$. The \lstinline{new} function takes $q$ and $modulo$ as input, and is used to make a new instantiation of \lstinline{Rq}.

\subsubsection{\lstinline{reduce}}

The \lstinline{reduce} method found in the quotient ring module is called on an \lstinline{Rq} struct, takes a polynomial \lstinline{pol} as input, and returns the normal form of the element \lstinline{pol} with respect to \lstinline{modulo}.

To achieve this, the method first performs polynomial long division with \lstinline{pol} as the dividend and the \lstinline{modulo} from the \lstinline{Rq} struct as the divisor. The remainder computed in this way is then stored in the variable $r$.

Lastly, we reduce the coefficients of the resulting polynomial $r$ modulo $q$, by using the \lstinline{modulo} method defined in the \lstinline{poly.rs} module, and then return the result.

\subsubsection{\lstinline{add}, \lstinline{sub}, \lstinline{times}, \lstinline{neg}, \lstinline{mul}}

The methods \lstinline{add}, \lstinline{sub}, \lstinline{times}, \lstinline{neg}, \lstinline{mul} are called on an \lstinline{Rq} struct.
These methods first use the addition, scalar multiplication, negation, and polynomial multiplication methods (or some combination thereof), as defined in the the \lstinline{poly.rs} module on the input.
Then, \lstinline{reduce} is called, and the result is returned.

\subsection{Public-key encryption scheme (encryption.rs)}

The encryption scheme, as usual, has three major components:

\begin{itemize}
  \item the \lstinline{generate_key_pair} function
  \item the \lstinline{encrypt} function
  \item and the \lstinline{decrypt} function
\end{itemize}

In addition to this we also have two functions responsible for the homomorphic operations, namely \lstinline{add} and \lstinline{mul}.

\subsubsection{\lstinline{generate_key_pair}}

The \lstinline{generate_key_pair} function takes as input an instance of the \lstinline{Parameters} struct.
This struct essentially just defines the parameters that nearly all functions in the encryption scheme use in some form or another.
This includes $r$, $N$, $q$, $p$, and the quotient ring $R_q$, which are all relevant for the key generation function.

The function starts by sampling polynomials $sk$ and $e_0$ from a Gaussian distribution with standard deviation $r$, and the polynomial $a_0$ uniformly from $R_q$.

It then calculates the public-key as $pk = (a_0, a_0 \cdot sk + e_0 \cdot p)$, and finally returns the key pair $(pk, sk)$.

\subsubsection{\lstinline{encrypt}}

The \lstinline{encrypt} function takes a \lstinline{Parameters} instance, as described above, and additionally takes a polynomial $m$ and a public key $pk$.

We extract the two polynomials of the public key, $a_0$ and $b_0$.

First, we make sure that the message polynomial we are trying to encrypt is in $R_p$.
Then, we sample polynomials $v$ and $e'$ from a Gaussian distribution with standard deviation $r$, and the polynomial $e''$ from a Gaussian distribution with standard deviation $r'$ (which is also defined in the \lstinline{Parameters} struct).

We then calculate $a = a_0 \cdot v + e' \cdot t$ and $b = b_0 \cdot v + e'' \cdot p$.
Finally, we create the ciphertext as a \lstinline{Vec} $c = [b + m, a]$ and return it.

\subsubsection{\lstinline{decrypt}}

The \lstinline{decrypt} function takes a \lstinline{Parameters} instance, as well as a ciphertext $c = [c_0, c_1, \dots]$ and a secret key $sk$.

We start by constructing the secret key vector $\mathbf{s} = [1, s, s^2, \dots]$ from the secret key.
We only create the first $|c|$ entries of the secret key vector, as those are the only ones we'll need.

We then initialize a polynomial $msg = 0$, which will become the decrypted message.
Then, iterating over each element $c_i$ in the ciphertext, we add $c_i \cdot \mathbf{s}_i$ to $msg$, where $\mathbf{s}_i$ is the $i$'th entry (zero-indexed) in the secret key vector.

Since the message has coefficients in $\mathbb{Z}_q$, but we need them to be in the interval from $-\frac{q}{2}$ to $\frac{q}{2}$, we call the \lstinline{normalized_coefficients} method on the $msg$ polynomial at this point.

At this point, we want to ensure that the $\ell_\infty$ for the message is at most $\frac{q}{2}$.
If not, decryption cannot work.

If the check succeeds, we reduce the message modulo $p$ to remove the $e \cdot p$ part of the encryption, and then we return the result.

\subsubsection{add}
The \lstinline{add} function homomorphically adds the two ciphertexts \lstinline{c1} and \lstinline{c2} given as input along with a \lstinline{Parameters} struct.

To do this computation the function simply creates a new \lstinline{Vec} of length \lstinline{max(c1.len(), c2.len()}.
Following this the function iterates over the two ciphertexts and for each entry $i$, it adds \lstinline{c1[i]} and \lstinline{c2[i]} to entry $i$ in the newly created \lstinline{Vec} using \lstinline{add} from \lstinline{quotient_ring.rs}. After iterating over all entries the resulting \lstinline{Vec} is returned.

\subsubsection{mul}
The \lstinline{mul} function homomorphically multiplies the two ciphertexts \lstinline{c1} and \lstinline{c2}, which it takes as input along with a \lstinline{Parameters} struct.

First, a new \lstinline{Vec} called \lstinline{res} is initialized.
For each entry $i$ in \lstinline{c1} and each entry $j$ in \lstinline{c2} the function adds $c1[i] \cdot c2[j]$ to the entry \lstinline{res[i + j]}, where the addition and multiplication is done using \lstinline{add} and \lstinline{mul} from \lstinline{quotient_ring.rs}.
Afterwards \lstinline{res} is returned.

\end{document}
