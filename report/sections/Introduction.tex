% !TeX root = ../main.tex
% Preamble
\documentclass[../main.tex]{subfiles}
\graphicspath{{\subfix{../images/}}}

\begin{document}
\section{Introduction}
	By the work of Peter W. Shar, we already know of an algorithm that is expected to break
	the prime factorization and discrete logarithm problems in polynomial time, assuming that it can be run
	on a large enough quantum computer. %TODO: indsæt lige en kommentar der linker til Shors algorithm - https://arxiv.org/abs/quant-ph/9508027
	And while quantum computers today are not large enough, the field is increasing rapidly and as such
	we should strive for new solutions.
	One solution is to change the underlying mathemathical problem to no longer use prime factorization or
	discrete log and instead use encryption schemes based on polynomial learning with error as proposed by %TODO: cite Lyubashevsky, Perikert and Regev
	which reduces down to the worst case problems on ideal lattices, making these encryption schemes a good
	candidate for future encryption schemes.
	Another highly desired property in cryptography is that of secure multiparty computation (MPC).
	In this problem n players wish to compute some function such that only the output of the function
	is revealed to every player, even in the case of an adversary controlling some of the corrupted players.
	It has been shown possible, for only a limited number of operations, by Brakerski and Vaikuntanathan
	%TODO: \cite{}
	and by Ivan et al. \cite{damgaard2012multiparty} that it is possible to get
	a homormorphic encryption scheme based on the polynomial learning with error problem and as such
	creating not only a scheme, thought to be secure against a quantum adversary, but also one which
	have the properties of MPC systems. % TODO: jeg kan ikke lide den her formulering.
	While encryption based on lattices seem very promising, parameters can still be chosen such that the
	underlying problem does not impose a desired level of security.
	To prevent this the lwe estimator tool 
	% TODO: \cite{}
	, maintained by Martin Albrecht, have been used to argue about the security of different parameters.
	\\[5mm]
	During this project we have implemented the system mostly inspired by the work of Brakerski and Vaikuntanathan (BV)
	% TODO: \cite{}
	and Ivan et al. \cite{damgaard2012multiparty}.
	Our system is made in the programming language rust, which provides absolute memory safety,
	making it an ideal choice for doing low level efficient implementations, while still having many of
	the nice features of a modern language.
	While performance has not been our main concern, we have tried to optimize the code written as much as possible,
	using and evaluating many of the features in the rust language, to have the optimal code for the features
	implemented.

\end{document}
