% !TeX root = ../main.tex
% Preamble
\documentclass[../main.tex]{subfiles}
\graphicspath{{\subfix{../images/}}}

\begin{document}

By the work of Peter W. Shor, we know of an algorithm that is expected to break
the prime factorization and discrete logarithm problems in polynomial time, assuming that it can be run
on a large enough quantum computer \cite{Shor_1997}.
While quantum computers today are not large enough, the field is increasing rapidly and as such
we should strive for new solutions.

One solution is to change the underlying mathemathical problem to no longer use prime factorization or
discrete log, and instead use encryption schemes based on polynomial learning with errors as proposed by \cite{IdealLatAndRLWE}. This problem can be reduced to worst case problems on ideal lattices, making it a good candidate for future encryption schemes, as can also be seen by the submissions to the post-quantum NIST competition, since many of the submissions are based on lattices.

Another highly desired property in cryptography is that of secure multiparty computation (MPC).
In this problem, $n$ players wish to compute some function such that only the output of the function
is revealed to every player, even in the case of an adversary controlling some of the corrupted players.
It has been shown, for a limited number of operations, by Brakerski and Vaikuntanathan \cite{brakerski2011fully} that it is possible to get a homomorphic encryption scheme based on the polynomial learning with errors problem.

While encryption based on lattices seems very promising, parameters can still be chosen such that the
underlying problem does not impose the desired level of security.
To prevent this, the LWE estimator tool like \cite{cryptoeprint:2015:046}, maintained by Martin Albrecht, can be used to argue about the security of different parameter sets.
\\[5mm]
During this project we have implemented a system, mostly inspired by the work of Brakerski and Vaikuntanathan (BV) \cite{brakerski2011fully} and Ivan et al. \cite{damgaard2012multiparty}.
Our system is made in the Rust programming language, which provides absolute memory safety,
making it an ideal choice for doing low-level efficient implementations, while still having many of
the nice features of a modern language.
While performance has not been our main concern, we have tried to optimize the code written when possible, using and evaluating many of the features in the Rust language to have the optimal code for the features implemented.

\end{document}
