\documentclass{article}
\usepackage[utf8]{inputenc}
\usepackage{amsmath}
\usepackage{amssymb}
\usepackage{listings}

% Code environment (incomplete)
\lstset{
  basicstyle=\ttfamily,
}

\begin{document}
\section{Quotient ring (\lstinline{quotient\_ring.rs})}
Encryption, decryption, and key generation involves adding, subtracting, multiplying, and negating elements in the quotient ring $R_q = Z_q[x]/\langle x^n + 1 \rangle$. To be able to do these computations we made a quotient ring implementation, which can be found in \lstinline{quotient_ring.rs}.

\subsection{\lstinline{Rq}}
The quotient ring module contains a struct definition \lstinline{Rq}, which represents an instantiation of a quotient ring $Z_q[x]/\langle f(x) \rangle$. It therefore has fields $q$ and $modulo$, where $q$ is an \lstinline{i128}, and $modulo$ is a \lstinline{Polynomial} representing $f(x)$. The \lstinline{new} function takes $q$ and $modulo$ as input, and is used to make a new instantiation of \lstinline{Rq}.

\subsection{\lstinline{reduce}}
The \lstinline{reduce} method found in the quotient ring module is called on an \lstinline{Rq} struct, takes a polynomial \lstinline{pol} as input, and returns the normal form of the element \lstinline{pol} with respect to \lstinline{modulo}.

To achieve this the method first does polynomial long division with \lstinline{pol} as the dividend and the \lstinline{modulo} from the \lstinline{Rq} struct as the divisor. The remainder computed in this way is then stored in the variable $r$.

Lastly, we reduce the coefficients of the resulting polynomial $r$ modulo $q$, by using the remainder operation($\%$) defined in the \lstinline{poly.rs} module, and then return the result.

\subsection{\lstinline{add}, \lstinline{times}, \lstinline{neg}, \lstinline{mul}}
The methods \lstinline{add}, \lstinline{times}, \lstinline{neg}, \lstinline{mul} are called on an \lstinline{Rq} struct. These methods first use the addition, scalar multiplication, negation, and polynomial multiplication methods defined in the the \lstinline{poly.rs} module on the input. Then a reduce call is done with the result as input to get a new $R_q$ element.

\end{document}
