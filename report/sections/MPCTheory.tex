\documentclass{article}
\usepackage[utf8]{inputenc}
\usepackage{amsmath}
\usepackage{amssymb}
\usepackage{listings}

% Code environment (incomplete)
\lstset{
  basicstyle=\ttfamily,
}

\begin{document}
\section{Multiparty Computation}
The Multiparty Computation (MPC) problem is the problem where $n$ players each with some private input $x_i$, want to compute some function $f$ on the input, without revealing anything but the result.
\section{Security of MPC protocols}
% Active vs passive
% \section{Homomorphic Encryption} Antager at det her allerede er dækket
\section{Bootstrapping}
% Turns SHE into FHE
\section{Protocol for MPC based on SHE}
In [] the authors describe a protocol for MPC based on a SHE scheme. The protocol is able compute arithmetic formulas consisting of up to a single multiplication, along with a relatively large number of additions, while being statistically UC-secure against an active adversary and $n - 1$ corruptions.

The protocol proceeds in two phases. In the first phase, preprocessing, a global key $[\![\alpha]\!]$, $n$ random values in two representations $[\![r]\!], \langle r \rangle$, and a number of multiplicative triples $\langle a \rangle, \langle b \rangle, \langle c \rangle$ satisfying $c = ab$ are generated. 

In the second phase, the online phase, the players use the global key and random values generated in the preprocessing phase to do the actual computations, aswell as a multiplicative triple if a multiplication is to be performed.

Thus the online phase only makes indirect use of the SHE scheme, as it is only used in the preprocessing phase to generate input for the online phase.

These two phases are described in further detail in section \ref{Prep} \& \ref{Online}.

\subsection{Abstract SHE scheme}
The cryptosystem used as the SHE scheme in the protocol has to have certain properties. In particular, such a cryptosystem consists of the algorithms (ParamGen, KeyGen, KeyGen*, Enc, Dec) which behave as follows
\paragraph{ParamGen}
\paragraph{KeyGen}
Outputs a keypair $pk, sk$.
\paragraph{KeyGen*} 
\paragraph{Enc}
\paragraph{Dec}

\subsection{Concrete instantiation of abstract SHE scheme}
The authors also present a concrete instantiation of the abstract SHE scheme described, namely based on the Ring-LWE based public key encryption scheme by Zvika Brakerski and Vinod Vaikuntanathan [] described in section x.x. The \textbf{KeyGen}, \textbf{Enc}, and \textbf{Dec} algorithms of the scheme behave precisely as as the key-gen, encryption, and decryption algorithms described in section x.x. % Der er små afvigelser (enc kan tage randomness som input, mens enc og dec ikke bruger -a, men a som c1). Ved ikke lige hvad vi gør ved det
The rest of the algorithms are then defined as follows
\paragraph{ParamGen}
\paragraph{KeyGen*} 

\subsection{Preprocessing phase} \label{Prep}
\subsection{Online phase} \label{Online}
\subsection{Parameter setting}
In section D of the paper the authors give example parameter sets and explain how to choose the parameters, s.t. the Ring-LWE public key encryption schemes meets the requirements of the MPC protocol.

\end{document}
