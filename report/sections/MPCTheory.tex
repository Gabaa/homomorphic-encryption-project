% !TeX root = ../main.tex
\documentclass[../main.tex]{subfiles}

\begin{document}

\section{Protocol for Multiparty Computation based on SHE} \label{section: MPC scheme}
The Multiparty Computation (MPC) problem is the problem where $n$ players each with some private input $x_i$, want to compute some function $f$ on the input, without revealing anything but the result.

In \cite{damgaard2012multiparty} the authors describe a protocol for MPC based on a SHE scheme. The protocol is able compute arithmetic formulas consisting of up to a single multiplication, along with a relatively large number of additions, while being statistically UC-secure against an active adversary and $n - 1$ corruptions.

The protocol proceeds in two phases. In the first phase, preprocessing, a global key $[\![\alpha]\!]$, random values in two representations $[\![r]\!], \langle r \rangle$, and a number of multiplicative triples $\langle a \rangle, \langle b \rangle, \langle c \rangle$ satisfying $c = ab$ are generated.

In the second phase, the online phase, the players use the global key and secret-shared data generated in the preprocessing phase to do the actual computations.

The online phase therefore only makes indirect use of the SHE scheme, as it is only used in the preprocessing phase to generate input for the online phase.

These two phases are described in further detail in section \ref{subsection: Prep} \& \ref{subsection: Online}.

\paragraph{Representations of shared values}
The protocol makes use of two different representations of shared values $[\![ \cdot ]\!], \langle \cdot \rangle$. For a shared value $a \in F_{p^k}$ the $\langle \cdot \rangle$ representation is defined as follows:
$$\langle a \rangle := (\delta, (a_1, ..., a_n), (\gamma(a)_1, ..., \gamma(a)_n))$$
where $a = \sum_i a_i$ and $\alpha (\delta + a) = \sum_i \gamma(a)_i$. The $\gamma(a)_i$ values are thus MAC values used to authenticate $a$. Such a value is shared s.t. each party $P_i$ has access to the global value $\delta$ along with shares $(a_i, \gamma(a)_i)$. Multiplication by a constant and regular addition and  are then defined entry-wise on the representation, while addition by a constant is defined as $$c + \langle a \rangle := (\delta - c, (a_1 + c, ..., a_n), (\gamma(a)_1, ..., \gamma(a)_n))$$
When a value $\langle a \rangle$ is partially opened, it means that the value $a$ is revealed without revealing $a$'s MAC values.

The second representation used for the protocol, $[\![ \cdot ]\!]$, is defined in the following way:
$$[\![ a ]\!] = ((a_1, ..., a_n), (\beta_i, \gamma(a)^i_1, ..., \gamma(a)^i_n)_{i = 1, ..., n})$$
where $a = \sum_i a_i$ and $a\beta_i = \sum_j \gamma(a)^j_i$. Thus the $\gamma(a)^j_i$ values are used to authenticate $a$ under $P_i$'s personal key $\beta_i$. Each player $P_i$ then has the shares $(a_i, \beta_i, \gamma(a)^i_1, ..., \gamma(a)^i_n)$. To open a $[\![ \cdot ]\!]$ value each player $P_j$ sends $a_j, \gamma(a)^j_i$ to $P_i$, who checks that $a\beta_i = \sum_j \gamma(a)^j_i$. Afterwards $P_i$ can compute $a = \sum_i a_i$.

\subsection{Abstract SHE scheme and instantiation}
The cryptosystem used as the SHE scheme in the protocol has to have certain properties to be admissible. The authors of the MPC protocol present a concrete instantiation of such an abstract SHE scheme, using the Ring-LWE based public key encryption scheme by Zvika Brakerski and Vinod Vaikuntanathan \cite{brakerski2011fully} described in section \ref{sec:RLWE-theory}.

\paragraph{encode and decode}
To be able to use this scheme we have to define the function \textbf{encode}, which maps elements in the plaintext space $M = (\mathbb{F}_{p^k})^s$ to elements in a ring $R$, which is equivalent to $\mathbb{Z}^N$. In addition to this we also need to define a function \textbf{decode}, which maps elements in $\mathbb{Z}^N$ to $M$, s.t. $\textbf{decode}(\textbf{encode}(m)) = m$ for $m \in M$.

To do this we first have to pick a polynomial for the quotient ring $R = \mathbb{Z}[X]/\langle f(x) \rangle$ used by the cryptosystem. Picking $f(x)$ in a specific way allows an optimization, which makes it possible to do entrywise multiplication, such that we can perform $N$ multiplications in parallel.
To do this we pick $f(x)$ to be the $m$'th cyclotomic polynomial $f(x)$ of degree $N = \phi(m)$, s.t. modulo $t$ the polynomial $f(x)$ factors into $l'$ irreducible factors of degree $k'$, where $l' \geq s$ and $k$ divides $k'$.

We can then define the function $\phi: M \rightarrow R_p$ which embeds $M$ into $R_p$. We also define $\iota : R_p \rightarrow \mathbb{Z}^N$, which maps the coefficients from the polynomial given as input to a vector of length $N$ with coefficients in the range $(-p/2, ..., p/2]$.

Finally, we define $\textbf{encode}(\textbf{m}) = \iota(\phi(\textbf{m}))$ and $\textbf{decode}(\textbf{x}) = \phi^{-1}(\textbf{x} \; (\text{mod } t))$.

\paragraph{key distribution and distributed decryption}
In addition to the aforementioned requirements, we also want the cryptosystem to implement a functionality $\mathcal{F}_{KeyGenDec}$. This functionality will on receiving ''start'' from all honest players generate a keypair $(pk, sk)$, and then distribute $pk$ to the players and store $sk$. The players can then use the functionality to cooperate in decrypting a ciphertext encrypted under $pk$.

\subsection{Preprocessing phase} \label{subsection: Prep}
The preprocessing phase is implemented by the Prep protocol, which consists of the steps \textbf{initialize}, \textbf{pair}, and \textbf{triple}. These steps use the additional protocols Reshare, PAngle, and PBracket as subroutines.

\paragraph{Protocol Reshare:} The Reshare protocol takes a ciphertext $e_m$ as input and a parameter $enc$, which can be set to either $NewCiphertext$ or $NoNewCiphertext$. The protocol then outputs a share $m_i$ of $m$ to each player along with a new fresh ciphertext $e'_m$ if $enc = NewCiphertext$, where $e'_m$ contains $\sum_i m_i$.
To do this the players first each sample $f_i \in \mathbb{F}_{p^k}$, and then broadcast $$e_{f_i} \leftarrow Enc_{pk}(f_i)$$
Each $P_i$ then runs the ZKPoPK protocol while acting as a prover on the previously generated ciphertext $e_{f_i}$, and if any of these proofs fail, then parties abort.
Now, each player homomorphically adds each encrypted share
$$e_f \leftarrow e_{f_1} \boxplus ... \boxplus e_{f_n} $$
and then homomorphically adds $e_m$ and $e_f$ to get $e_{m + f}$. The players now use $\mathcal{F}_{KeyGenDec}$ to decrypt $e_{m + f}$ so that they get $m + f$. Now $P_1$ sets $m_1 \leftarrow m + f - f_1$, while the rest of the players $P_i$ set $m_i \leftarrow - f_i$.
If $enc = NewCiphertext$, then the players each compute
$$e'_m \leftarrow Enc_{pk}(m + f) \boxminus e_{f_1} \boxminus ... \boxminus e_{f_n}$$ where default randomness is used for the encryption.

\paragraph{Protocol PAngle:}
PAngle takes as input a ciphertext $e_v$ along with privately held shares $v_1, ..., v_n$. These are then used to generate a value in the angle representation $\langle v \rangle$.
To achieve this all players first compute $$e_{v \cdot a} \leftarrow e_v \boxplus e_\alpha$$
Reshare is then used with $e_{v \cdot \alpha}$ as input, such that each player $P_i$ recieves a share $\gamma_i$ of $v \cdot \alpha$. Finally, $\langle v \rangle = (0, (v_1, ..., v_n), (\gamma_1, ..., \gamma_n)$ is output.

\paragraph{Protocol PBracket:}
PAngle takes as input a ciphertext $e_v$ along with privately held shares $v_1, ..., v_n$. These are then used to generate a value in the angle representation $[\![ v ]\!]$.
For $i = 1, ..., n$ all players compute $$e_{\gamma_i} \leftarrow e_{\beta_i} \boxtimes e_v$$ and then generate $(\gamma^1_i, ..., \gamma^n_i)$ by calling Reshare with $e_{\gamma_i}$ and $NoNewCiphertext$ as input.
The representation $[\![ v ]\!] = ((v_1, ..., v_n), (\beta_i, \gamma(v)^i_1, ..., \gamma(v)^i_n)_{i = 1, ..., n})$ is then output.

\paragraph{Initialize:}
The \textbf{initialize} step generates the global and personal keys. This is acheived by the players first calling "start" on $\mathcal{F}_{KeyGenDec}$, so that every player obtains the public key $pk$.
Then each player samples $\alpha_i, \beta_i \in \mathbb{F}_{p^k}$, and broadcasts $$e_{\alpha_i} \leftarrow Enc_{pk}(Diag(\alpha_i)), \;\; e_{\beta_i} \leftarrow Enc_{pk}(Diag(\beta_i))$$ where $Diag(a) = (a, a, ..., a) \in (\mathbb{F}_{p^k})^s$. Now each $P_i$ runs the ZKPoPK protocol twice with $diag = true$, while acting as a prover, where the inputs are the ciphertexts $e_{\alpha_i}$ and $e_{\beta_i}$ repeated $sec$ times.
Finally, the players homomorphically add the encrypted shares $e_{\alpha_i}$ to get $e_\alpha$, which they use along with their share $Diag(\alpha_i)$ to generate $[\![ Diag(\alpha) ]\!]$ using a call to PBracket. Then $[\![ Diag(\alpha) ]\!]$ is the global key, while $\beta_i$ is $P_i$'s personal key.

\paragraph{Pair:}
In \textbf{pair} the players generate random values in the two representations $[\![ r ]\!], \langle r \rangle$. This is done by each player first sampling a share $r_i \in (\mathbb{F}_{p^k})^s$, then broadcasting Each player then encrypts their share to get $$e_{r_i} \leftarrow Enc_{pk}(r_i)$$ which they then broadcast. Once again $P_i$ will now runs the ZKPoPK protocol acting as a prover with input $e_{r_i}$, and if the ZK proof fails, then the protocol is aborted.
The players then homomorphically add the encrytped shares to get $e_r$, which they use along with their share $r_i$ as input to PBracket and PAngle to generate $[\![ r ]\!], \langle r \rangle$.

\paragraph{Triple:}
The \textbf{triple} step generates triples $(\langle a \rangle, \langle b \rangle, \langle c \rangle)$ satisfying $c = ab$. To do this the players start off by sampling shares $a_i, b_i \in (\mathbb{F}_{p^k})^s$.
The players then encrypt their shares and broadcast the result $$e_{a_i} \leftarrow Enc_{pk}(a_i), \;\; e_{b_i} \leftarrow Enc_{pk}(b_i)$$
Now each $P_i$ acts as a prover running the ZKPoPK protocol first with $e_{a_i}$ and then with $e_{b_i}$ as input, and if any proof fails, then the protocol is aborted.
The players then homomorphically add the encrypted shares to get $e_a$ and $e_b$, and use these along with their shares $a_i, b_i$ to generate $\langle a \rangle$, $\langle b \rangle$ using calls to PAngle.
Following this each player homomorphically multiplies $e_a$ and $e_b$ to get $e_c$, which the players use as input to Reshare to get shares of $c$ along with a new ciphertext $(c_1, ..., c_n, e_{c'})$. Then the players then use their shares of $c$ along with $e_{c'}$ to generate $\langle c \rangle$ by calling PAngle. Finally, the triple $(\langle a \rangle, \langle b \rangle, \langle c \rangle)$ is output.


\subsection{Online phase} \label{subsection: Online}
The Online protocol implements the online phase,
and consists of the steps \textbf{initialize}, \textbf{input}, \textbf{add}, \textbf{multiply}, and \textbf{output}. These steps are executed as needed to evaluate the arithemtic circuit that we wish to evaluate.

\paragraph{Initialize:} The \textbf{initialize} step simply consists using the Prep protocol to generate a global key, along with enough multiplicative triples and random values in the two representations showed earlier, for the circuit that we want to evaluate.

\paragraph{Input:} The \textit{input} step lets a player $P_i$ share their private input $x_i$. The input is shared by taking a pair $[\![ r ]\!], \langle r \rangle$, and then opening $[\![ r ]\!]$ to $P_i$ so that $P_i$ gets $r$. Following this $P_i$ computes and broadcasts $\epsilon \leftarrow x_i - r$. All players finally set $\langle x_i \rangle \leftarrow \langle r \rangle + \epsilon$. %Forklar hvordan en konstant adderes til angle repr.

\paragraph{Add:} To add two values $\langle x \rangle, \langle y \rangle$, we simply perform the component-wise addition $\langle z \rangle = \langle x \rangle + \langle y \rangle$, meaning that each player adds their shares locally $z_i = x_i + y_i, \gamma(z)_i = \gamma(x)_i + \gamma(y)_i$. %Er det klart hvad der sker her?

\paragraph{Multiply:} To multiply two values $\langle x \rangle, \langle y \rangle$, we use two multiplicative triples $(\langle a \rangle, \langle b \rangle, \langle c \rangle)$ and $(\langle f \rangle, \langle g \rangle, \langle h \rangle)$. We use the second triple to check that $ab = c$, but this could also be done in preprocessing instead. To do this check we first open $[\![ t ]\!]$ to get t, then partially open $t * \langle a \rangle - \langle f \rangle$ and $\langle b \rangle - \langle g \rangle$ to get $\rho$ and $\sigma$ respectively. Finally, we compute and partially open $$t * \langle c \rangle - \langle h \rangle - \sigma \langle f \rangle - \rho \langle g \rangle - \sigma \rho$$
if the result is $0$, then we conclude that $ab = c$ and move on, otherwise the players abort.

Now, partially open $\langle x \rangle - \langle a \rangle$ and $\langle y \rangle - \langle b \rangle$ to get $\epsilon$ and $\delta$ respectively. Finally, we ouptut $$\langle z \rangle \leftarrow \langle c \rangle + \epsilon \langle b \rangle + \delta \langle a \rangle + \epsilon \delta$$

\paragraph{Output:}
To output a value y given $\langle y \rangle$, the players do the following.
First, a random value $[\![ e ]\!]$ is opened so each player gets $e$, which is used to compute $$a = \sum_j e^j \cdot a_j$$ where $a_j$ are all of the opened values of the form $\langle a_j \rangle$.
Each player $P_i$ then uses the commitment functionality $\mathcal{F}_{Com}$ to commit to $\gamma_i \leftarrow \sum_j e^j \gamma(a_j)_i$ along with $y_i$ and $\gamma(y)_i$.
Then, the global key $[\![ \alpha ]\!]$ is opened.
The players then use $\mathcal{F}_{Com}$ to open $\gamma_i$ and then check that indeed $$\alpha(a + \sum_j e^j \delta_j) = \sum_j \gamma_i$$
If this is not the case, then the players abort.
Finally, to make sure that all players end up with $y$ the commitments to $y_i$ and $\gamma(y)_i$ are opened. Now the players check that $\alpha(y + \delta) = \sum_i \gamma(y)_i$, and if this is the case, then $y = \sum_i y_i$ is output.


%\subsection{Parameter setting}
%In section D of the paper the authors give example parameter sets and explain how to choose the parameters, s.t. the previously described Ring-LWE public key encryption scheme meets the requirements of the MPC protocol.

\section{Reuse of unrevealed secret-shared data} \label{section: Reuse}
In \cite{damgaard2013practical} a technique that allows for reuse of unrevealed secret-shared data is used.
This technique revolves around not having to open $[\![ \alpha ]\!]$, and in fact we do not even need the $[\![ \cdot ]\!]$ representation when using this technique.

The technique works as follows.
First, when we generate the global key we now just need each player $P_i$ to have a share $\alpha_i$ of $\alpha$. In the output step the player will then instead of the current MAC check instead invoke the new MACCheck protocol on all of the values in the $\langle \cdot \rangle$ representation that have been opened, and if this succeeds the player invokes MACCheck on the output value $\langle y \rangle$. The MACCheck protocol works as follows:
\paragraph{Protocol MACCheck}
First each $P_i$ samples a seed $s_i$ and use $\mathcal{F}_{Commit}$ to broadcast $\tau^s_i \leftarrow Commit(s_i)$. Following this each player opens all commitments using $\mathcal{F}_{Commit}$ to get all $n$ seeds $s_j$. Now, all players set $$s \leftarrow s_1 \oplus ... \oplus s_n$$
Players then use $s$ as seed to sample a random vector of length $t$ with entries in the interval $[0, p)$. %WAT, t eller p
All players then first compute $$a \leftarrow \sum^t_{j = 1} r_j \cdot a_j$$ where the $a_j$'s are the opened values.
Now $P_i$ computes
\begin{align*}
    \gamma_i \leftarrow \sum^t_{j = 1} r_j \cdot \gamma(a_j)_i \text{  and  } \sigma_i \leftarrow \gamma_i - \alpha_i \cdot a
\end{align*}
Player $i$ then uses $\mathcal{F}_{Commit}$ to broadcast $\tau^\sigma_i \leftarrow Commit(s_i)$.
All players invoke $\mathcal{F}_{Commit}$ to open the commitments received to get the $\sigma_j$'s.
Finally, the players check that $\sigma_1 + ... + \sigma_n = 0$, and if this is not the case, then they abort.

\section{Zero-knowledge proof}
    In \cite{damgaard2012multiparty} a zero knowledge proof of plaintext knowledge is shown in figure 9 and 10.
    We have implemented the zero knowledge from figure 10, as they the authors state that it will be better suited for implementation.
    % TODO: skal vi skrive det eller kan vi bare smide et billede ind? også forklare hvad det er for nogle variabler der bliver sat?

\section{Choosing parameters for our MPC system} \label{sec:choosing-parameters}
    When choosing our parameter sets for the MPC system it is important to think not only of the security but also of
    the practicality of the systems.
    In the article \cite{damgaard2012multiparty} from Ivan et al.
	We are provided with certain limitations on the parameters as well, which we have to bend a bit to test that the system works and as such we will sketch the calculations for the parameters here.
    We will omit the proofs and justification as those are outside the scope of our project.
    Let N be the degree of our polynomial f(x) in the RLWE encryption scheme, % TODO: cite the section.
    let r be the randomness used while encrypting, SEC be 40, the amount of players n be 3 with $c_{SEC}, Y$ and $Z$ be
    be given by the formulas
    \begin{align*}
        c_{sec} &= 9 \cdot N^2 \cdot SEC^4 \cdot 2^{SEC + 8} \\
        Y       &= \frac{p}{2} + p \cdot (4 \cdot C_{m-bound} \cdot r^2 \cdot N^2 \cdot 2 \cdot \sqrt{N} \cdot r + 4 \cdot C_{m-bound} \cdot r^2 \cdot N^2) \\
        Z       &= C_{m-bound} \cdot N^2 \cdot n^2 \cdot c_{sec}^2 \cdot Y^2 + n \cdot c_{sec} \cdot Y
    \end{align*}
    Then the article writes that the following inequalities for q and r need to be held.
    \begin{align*}
        q   &> 2 \cdot Z \cdot (1 + 2^{sec}) \\
        r   &> max\{3.2, 1.5 \cdot \gamma^{-t'} \cdot q^{1 - \frac{N}{t'}}\}
    \end{align*}
    We wrote a small program, which can be seen below, to check for good values of q and r.
    \lstinputlisting[language=Python]{code-snippets/param-searach.py}
	With the script we found that a  512 degree polynomial would result in q being of 313 bits in size and our randomness (used for encryption in the mpc protocol) should be above $3.85 \cdot 10^{73}$.
	While these values provides something of a pretty good security, the generated value for the randomness r is way to high to be
	practical in any way.
	So to have a smaller degree of the polynomial f(x), and have better performance,
	we need to increase the value of q and r.
	Using this enormous value of r, we would be in a situation where the l-inf-norm would be to high, which
	would generate to much noise for us to decrypt and as a consequence, we need to have a smaller value of r.
	\\[5mm]
    By using the tool from Martin Albrecht % TODO: indsæt lige en reference.
    and using the valus which we got from the program as above, we can see that setting q some prime
    of size 313 bits and setting the degree of our polynomial to N = 12900, we will have around 627 bits
    of security against the BKZ lattice reduction.
    These tests also align with the how the parameters in \cite{damgaard2012multiparty} are chosen.
    \\[5mm]
    While this provides pretty good security, it is by no means effective for us to run, as we miss the optimization %TODO: indsæt ref
    so we have to settle for a smaller degree but still leave q high enough to not have enough noise, when 
    we perform the protocol operations.
    From an empirical analysis we cannot run with a polynomial f(x) having a degree > 512 and we still have to provide a q that is
    around the 300 bits in size.
    Running these parameters through the lwe-estimator, we can see that our security crashes all the way
    down to around 13 bits. % TODO: vigtig pointe her, skal med konklusionen!

\end{document}
