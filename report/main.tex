\documentclass[11pt,openright]{report}

\usepackage[utf8]{inputenc}
\usepackage[american]{babel}
\usepackage{a4}
\usepackage{url}
\usepackage{latexsym}
\usepackage{amssymb}
\usepackage{amsmath}
\usepackage{epsfig}
\usepackage[T1]{fontenc}
\usepackage{mathptmx}
\usepackage{color}
\usepackage{epstopdf}
\usepackage{microtype}
\usepackage{hyperref}
\usepackage[useregional]{datetime2}
\DTMlangsetup[en-US]{showdayofmonth=false}
\usepackage{listings}
\usepackage{placeins}
\usepackage{graphicx}
\usepackage{subcaption}
\usepackage{mathtools}
\usepackage{todonotes}
\usepackage{amsthm}

\graphicspath{{images/}{../images/}}
\usepackage{subfiles} % Best loaded last in the preamble
	
\renewcommand*\sfdefault{lmss}
\renewcommand*\ttdefault{txtt}
\theoremstyle{definition}
\newtheorem*{definition}{Definition}

\begin{document}

%%%%%%%%%%%%%%%%%%%%%%%%%%%%%%%%%%%%%%%%%%%%%%%%%%%%%%%%%%%%%%%%%%%%%%%

\pagestyle{empty}
\pagenumbering{roman}
\vspace*{\fill}\noindent{\rule{\linewidth}{1mm}\\[4ex]
{\Huge\sf Multiparty Computation based on Ring-LWE}\\[4ex] %Bare midlertidig titel
{\huge\sf Mikkel Gaba, Marcus Sellebjerg, Kasper Ils{\o}e}\\[2ex]
\noindent\rule{\linewidth}{1mm}\\[4ex]
\noindent{\Large\sf Project Report (10 ECTS) in Computer Science\\[1ex]
  Advisor: Ivan Damg{\aa}rd \\[1ex]
  Department of Computer Science, Aarhus University\\[1ex]
  May 22th, 2022 \\[15ex]}\\[\fill]}
\epsfig{file=logo.eps}\clearpage
\linespread{1.15}

%%%%%%%%%%%%%%%%%%%%%%%%%%%%%%%%%%%%%%%%%%%%%%%%%%%%%%%%%%%%%%%%%%%%%%%

%%% Code environment
\definecolor{commentsColor}{rgb}{0.497495, 0.497587, 0.497464}
\definecolor{keywordsColor}{rgb}{0.000000, 0.000000, 0.635294}
\definecolor{stringColor}{rgb}{0.558215, 0.000000, 0.135316}


\lstset{
  basicstyle=\ttfamily\small,                   % the size of the fonts that are used for the code
  breakatwhitespace=false,                      % sets if automatic breaks should only happen at whitespace
  breaklines=true,                              % sets automatic line breaking
  frame=tb,                                     % adds a frame around the code
  commentstyle=\color{commentsColor}\textit,    % comment style
  keywordstyle=\color{keywordsColor}\bfseries,  % keyword style
  stringstyle=\color{stringColor},              % string literal style
  numbers=left,                                 % where to put the line-numbers; possible values are (none, left, right)
  numbersep=5pt,                                % how far the line-numbers are from the code
  numberstyle=\tiny\color{commentsColor},       % the style that is used for the line-numbers
  showstringspaces=false,                       % underline spaces within strings only
  tabsize=2,                                    % sets default tabsize to 2 spaces
  language=Java
}


\pagestyle{plain}
\chapter*{Abstract}
\addcontentsline{toc}{chapter}{Abstract}

\subfile{sections/Abstract}

\vspace{2ex}
\begin{flushright}
  \emph{Mikkel Gaba, Marcus Sellebjergen, Kasper Ils{\o}e}\\
  \emph{Aarhus, May 22th, 2022.}
\end{flushright}

\tableofcontents
\cleardoublepage
\pagenumbering{arabic}
\setcounter{secnumdepth}{2}

%%%%%%%%%%%%%%%%%%%%%%%%%%%%%%%%%%%%%%%%%%%%%%%%%%%%%%%%%%%%%%%%%%%%%%%

\chapter{Introduction}
\label{ch:intro}

\subfile{sections/Introduction}


%%%%%%%%%%%%%%%%%%%%%%%%%%%%%%%%%%%%%%%%%%%%%%%%%%%%%%%%%%%%%%%%%%%%%%%

\chapter{Review of literature}
\label{ch:main1}

\subfile{sections/RingLwePKSystem}

\subfile{sections/CircuitPrivacy}

\subfile{sections/MPCTheory}


\chapter{Implementation}
\label{ch:main2}

\subfile{sections/ImplementationIntro}

\subfile{sections/PKCImplementation}

\subfile{sections/MPCImplementation}

\chapter{Testing}
\label{ch:main3}

\subfile{sections/ImplementationTesting}

%%%%%%%%%%%%%%%%%%%%%%%%%%%%%%%%%%%%%%%%%%%%%%%%%%%%%%%%%%%%%%%%%%%%%%%

\chapter{Conclusion}
\label{ch:conclusion}

\subfile{sections/Conclusion}

\chapter*{Acknowledgments}
\addcontentsline{toc}{chapter}{Acknowledgments}

\subfile{sections/Acknowledgments}

%%%%%%%%%%%%%%%%%%%%%%%%%%%%%%%%%%%%%%%%%%%%%%%%%%%%%%%%%%%%%%%%%%%%%%%

\cleardoublepage
\addcontentsline{toc}{chapter}{Bibliography}
\bibliographystyle{plain}
\bibliography{refs}

%%%%%%%%%%%%%%%%%%%%%%%%%%%%%%%%%%%%%%%%%%%%%%%%%%%%%%%%%%%%%%%%%%%%%%%

\cleardoublepage
\appendix
\chapter{First appendix}

\section{MPC system performance with polynomial long division}

\begin{table}[h]
  \centering
  \begin{tabular}{l|l}
      $N$  & Time (s)   \\ 
      \hline
      8   & 19.036512 \\
      16  & 39.95138 \\
      32  & 85.003136 \\
      64  & 200.78773 \\
      128 & 568.9214 \\
      256 & 1631.5626 \\
      512 & 5160.727
  \end{tabular}
  \caption{A table relating the size of the $N$ parameter to the wallclock run-time of the entire MPC system, for $n = 3$ with the function $x_1 \cdot x_2 + x_3$ when using polynomial long division.}
  \label{tab:polynomial-long-division-performance}
\end{table}

\end{document}

